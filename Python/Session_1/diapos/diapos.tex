\documentclass[11pt,aspectratio=169]{beamer}
\usetheme{CambridgeUS}
\beamertemplatenavigationsymbolsempty
\usepackage{fontspec}
\setsansfont{Junicode}
\usepackage{amsmath,mathtools}
\usepackage{hyperref}
\usepackage{graphicx}
\usepackage{xpatch}
\usepackage[export]{adjustbox}
\usepackage{pgfplots}
\usepackage{tikz}
\usetikzlibrary{shapes,calc,matrix,decorations.markings,decorations.pathreplacing,positioning, intersections,backgrounds,through,hobby}
\usepackage{csquotes}
\usepackage[french]{babel}
\date[2 octobre 2023]{2 octobre 2023}
\author[Matthias \textsc{Gille Levenson}]{\\~\\ Matthias \textsc{Gille Levenson}\\   {\scriptsize AMU -- CIHAM UMR 5648}\\ {\tiny prénom [point] gille [point] levenson [at] ens-lyon.fr}\vspace{-1cm}}
\title[Transcription automatisée et HTR]{Python pour le traitement des sources XML}
\titlegraphic{\includegraphics[scale=0.23]{/home/mgl/Bureau/Travail/admin/logos/logo-ciham.png}}


%\usepackage[labelformat=empty]{caption}
\setbeamertemplate{caption}{\insertcaption\par}

\usepackage[datamodel=thesis,citestyle=authoryear,isbn=true,
doi=true,backend=biber,language=french,url=true,sorting=nty,maxnames=6, maxcitenames=4]{biblatex}
\renewbibmacro{in:}{}
\renewcommand*{\bibfont}{\tiny} 
% http://mcclinews.free.fr/latex/introbeamer/elements_contenu.html
%\xapptobibmacro{cite}{\setunit{\nametitledelim}\printfield{year}}{}{}
\addbibresource{biblio.bib}

\begin{filecontents*}{thesis.dbx}
\ProvidesFile{thesis.dbx}[2014/06/14 supervisor for theses]
\RequireBiber[3]
\DeclareDatamodelFields[type=list,datatype=name]{supervisor}
\DeclareDatamodelEntryfields[thesis]{supervisor}
\end{filecontents*}


\begin{filecontents*}{french-thesis.lbx}
\ProvidesFile{french-thesis.lbx}[2014/06/14 english for thesis]
\InheritBibliographyExtras{french}
\NewBibliographyString{supervision,jointsupervision}
\DeclareBibliographyStrings{%
inherit           = {french},
supervision       = {{dirigée par}{dir\adddotspace }},
jointsupervision  = {{codirigée par}{codir\adddotspace }},
}
\end{filecontents*}

\DeclareLanguageMapping{french}{french-thesis}

\newbibmacro*{thesissupervisor}{%
  \ifnameundef{supervisor}{}{%
    \ifnumgreater{\value{supervisor}}{1}
      {\bibstring{jointsupervision}}
      {\bibstring{supervision}}
    \printnames{supervisor}}}

\xpatchbibdriver{thesis}
  {\printfield{type}}
  {\printfield{type}
   \newunit
   \usebibmacro{thesissupervisor}}
  {\typeout{yep}}
  {\typeout{no}}
  
% https://tex.stackexchange.com/a/184878 ajout direction thèse





\AtBeginSection[]
{\begin{frame}
 \frametitle{}  
 \tableofcontents[currentsection,
                  hideothersubsections,
                  subsubsectionstyle=show/show/show/hide
                   ]
 \end{frame} 
 }



\setbeamertemplate{sections/subsections in toc}[square]
\setbeamertemplate{bibliography item}[sqare]
\setbeamertemplate{itemize item}[square]
\setbeamertemplate{enumerate item}[square]
\setbeamertemplate{itemize subitem}[square]





\renewcommand*{\dotFFN}{}
\newcommand{\astfootnote}[1]{%
\let\oldthefootnote=\thefootnote%
\setcounter{footnote}{0}%
\renewcommand{\thefootnote}{\fnsymbol{footnote}}%
\footnote{#1}%
\let\thefootnote=\oldthefootnote%
}








\begin{document}
\maketitle



\section{Plan de la séance}


\begin{frame}
\begin{itemize}
\item Une partie de présentation technique, méthodologique et philologique
\item Une partie de présentation de l'interface
\item Une partie de travaux pratiques avec entraînement d'un modèle
\end{itemize}
\end{frame}


\begin{frame}{lxml et la non-récursivité}
\begin{itemize}
\item Le traitement des données XML via python ne suit pas la logique d'imbrication du XML
\item Il y aura donc des cas où python ne sera pas adapté (en général, tout ce qui a trait à la production d'éditions et à la transformation complexe de sources XML)
\item Python sera bien adapté au traitement complexe de documents XML qui nécessitent l'utilisation de listes, dictionnaires
\item De même, l'intérêt principal de Python est la possibilité d'utiliser des librairies et outils externes de traitement (traitement de l'image, TAL).
\end{itemize}
\end{frame}

\begin{frame}{Les espaces de nom}
\begin{itemize}
\item 
\end{itemize}
\end{frame}


\begin{frame}{Parser l'arbre XML}
\begin{itemize}
\item La méthode `parse()`
\item Dans le cas d'inclusions, on doit explicitement demander à ce que les sources externes soient aussi parsées
\end{itemize}
\end{frame}

\begin{frame}{Naviguer dans l'arbre}
\begin{itemize}
\item 
\end{itemize}
\end{frame}

\begin{frame}{Créer un élément}
\begin{itemize}
\item 
\end{itemize}
\end{frame}

\begin{frame}{Créer un attribut}
\begin{itemize}
\item 
\end{itemize}
\end{frame}





%\begin{frame}
%\frametitle{Plan} % Table of contents slide, comment this block out to remove it
%\tableofcontents % Throughout your presentation, if you choose to use \section{} and \subsection{} commands, these will automatically be printed on this slide as an overview of your presentation
%\end{frame}



\begin{frame}[allowframebreaks]{Références}
\printbibliography[keyword=dataset, title={Jeux de données}]
\hrule
\printbibliography[notkeyword=dataset, title={Sources secondaires}]
\end{frame}

\end{document}
